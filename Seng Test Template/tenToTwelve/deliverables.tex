This section consists in enumerating with a brief description all what we have to deliver after testing the Android module which involves the “USER INTERFACE OF THE RESEARCH SUPPORT SYSTEM (RSS)”.  \\*

The deliverables of this test will be as follow:
\begin{itemize}
        \item The Test plan document which is the document containing all the aspects of the testing; it describes the different activities that was done during the testing as well as the scope of testing , and the approach we had to take in testing the UI of the android module of the RSS.
        \item The Test cases: It is based on the functional and non-functional requirements.  The functional requirement consists of using mock object of Junit to apply the unit testing; and Android lint report is used for the quality requirement (non-functional requirement).

        \item Tools and their output: Two tools are used in the testing process. Robolectric Android which is a framework for unit testing that uses the Android SDK jar in order to test-drive the android module and  Android Lint Report for non-functional requirement.

        \item Errors logs and executions logs: are some screenshots of files that records some errors that occurs during the testing.

        \item Problem reports and corrective action: a corrective action was applied in the graddle file (file that manages dependencies in android studio) when opening the source code.     
\end{itemize}

\\*

NB: the only thing that will not be part of the test deliverables is the Android module.                                                                              