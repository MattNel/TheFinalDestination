\subsection{Testing Level -- Unit testing}
In the given code, no testing constructs were found, as such the Final Destination testing team will carry out the necessary tests.
Unit tests will be automated by the use of established Android UI testing frameworks. At this level of testing, mock objects will be required to ensure that the testing process can be carried out.

%Semaka
\subsection{Test Tools}
A few tools will be used to test the android module, namely: \\
\renewcommand{\labelenumi}{\Roman{enumi}}
\begin{itemize}
	\item Lint
	  Lint is a tool that checks the quality of your code and find errors that were missed during coding. It is a good tool to use as it can be configured to test code at different levels, i.e. test the modules separately or the entire project, as the developers and testers see fit. The given alpha android code will be tested individually to see if each of the given files work separately and then the it will also be tested on entire project to see how the different files work together.
	\item junit
	  JUnit belongs to the xUnit family of unit testing frameworks and it allows writing of repeatable tests for Java code.
\end{itemize}

%Semaka
\subsection{Meetings}
The team will meet at least once a week to discuss progress so far and to divide up the remaining work load.

\subsection{Measures and Metrics}
Metrics for the testing of non-functional requirements will be collected. These will be made available by the Lint Report tool.

%Semaka
\subsection{Configuration Management}
Only Unit testing will be performed thus only one machine is needed. All testing will be performed using Android Studio. There are no specified hardware requirements for this module so any electronic device that has Android Studio installed and can run Java code can be used to test this module.
No modifications will be made to the given code except those that will help with testing so there is no need to create different branches on github so as not to hinder production. Thus, only master will be used to make things simpler for the testers.

\subsection{Special Requirements}
The Android module will be assessed according to the given evaluation criteria.

\begin{enumerate}

	\item Front-end process demo
	 \begin{description}
			\item The UI of the module should be representative of an Android application UI and should conform to the set standards.
		\end{description}
		
		\item Usability
	 \begin{description}
			\item With each click of a link or a button a user should be given appropriate feedback, in that a result of that click should be displayed or an appropriate error message should be thrown. The feedback should be timely as well.
		\end{description}

	\item Use of external mock objects
	 \begin{description}
			\item As a result of a modular system design, the Android module should be able to fulfil all requirements in isolation. As such the use of mock objects is suited as a substitute for the real objects provided by the other system modules. REST API mock objects are needed for this module.
			\item The implementation will be checked for the use of a Mock REST API.
		\end{description}
		
		 
		\item Gradle build with appropriate dependencies
	 \begin{description}
			\item Dependency management of the module should be handled with Gradle to allow for seamless project build. All Android dependencies should be kept in the Gradle configuration.
		\end{description}
		
	\item  Use of Dependency Injection
		\begin{description}
			\item As previously mentioned, the specification for the RSS requiress a modular design, ensuring for less inter-module dependencies. Each isolated module is meant to use Mock objects in place of concrete objects from the respective system modules. These mock objects are meant to be injected into the Android module.
		\end{description}


	\item Use of Fragments
	 \begin{description}
			\item The 
		\end{description}
		
	\item Use of REST
	 \begin{description}
			\item The Android module should interact with a REST API in order to receive the necessary objects to be displayed on the UI. REST http requests should be made depending on the required data and a back-end server should respond with the appropriate data as long as no error occurs.
		\end{description}

\end{enumerate}
